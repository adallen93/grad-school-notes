% Options for packages loaded elsewhere
\PassOptionsToPackage{unicode}{hyperref}
\PassOptionsToPackage{hyphens}{url}
%
\documentclass[
]{article}
\usepackage{amsmath,amssymb}
\usepackage{iftex}
\ifPDFTeX
  \usepackage[T1]{fontenc}
  \usepackage[utf8]{inputenc}
  \usepackage{textcomp} % provide euro and other symbols
\else % if luatex or xetex
  \usepackage{unicode-math} % this also loads fontspec
  \defaultfontfeatures{Scale=MatchLowercase}
  \defaultfontfeatures[\rmfamily]{Ligatures=TeX,Scale=1}
\fi
\usepackage{lmodern}
\ifPDFTeX\else
  % xetex/luatex font selection
\fi
% Use upquote if available, for straight quotes in verbatim environments
\IfFileExists{upquote.sty}{\usepackage{upquote}}{}
\IfFileExists{microtype.sty}{% use microtype if available
  \usepackage[]{microtype}
  \UseMicrotypeSet[protrusion]{basicmath} % disable protrusion for tt fonts
}{}
\makeatletter
\@ifundefined{KOMAClassName}{% if non-KOMA class
  \IfFileExists{parskip.sty}{%
    \usepackage{parskip}
  }{% else
    \setlength{\parindent}{0pt}
    \setlength{\parskip}{6pt plus 2pt minus 1pt}}
}{% if KOMA class
  \KOMAoptions{parskip=half}}
\makeatother
\usepackage{xcolor}
\usepackage[margin=1in]{geometry}
\usepackage{graphicx}
\makeatletter
\def\maxwidth{\ifdim\Gin@nat@width>\linewidth\linewidth\else\Gin@nat@width\fi}
\def\maxheight{\ifdim\Gin@nat@height>\textheight\textheight\else\Gin@nat@height\fi}
\makeatother
% Scale images if necessary, so that they will not overflow the page
% margins by default, and it is still possible to overwrite the defaults
% using explicit options in \includegraphics[width, height, ...]{}
\setkeys{Gin}{width=\maxwidth,height=\maxheight,keepaspectratio}
% Set default figure placement to htbp
\makeatletter
\def\fps@figure{htbp}
\makeatother
\setlength{\emergencystretch}{3em} % prevent overfull lines
\providecommand{\tightlist}{%
  \setlength{\itemsep}{0pt}\setlength{\parskip}{0pt}}
\setcounter{secnumdepth}{-\maxdimen} % remove section numbering
\ifLuaTeX
  \usepackage{selnolig}  % disable illegal ligatures
\fi
\IfFileExists{bookmark.sty}{\usepackage{bookmark}}{\usepackage{hyperref}}
\IfFileExists{xurl.sty}{\usepackage{xurl}}{} % add URL line breaks if available
\urlstyle{same}
\hypersetup{
  pdftitle={BIOSTAT 704 - Homework 2},
  pdfauthor={Austin Allen},
  hidelinks,
  pdfcreator={LaTeX via pandoc}}

\title{BIOSTAT 704 - Homework 2}
\author{Austin Allen}
\date{February 6, 2024}

\begin{document}
\maketitle

\hypertarget{problem-1}{%
\subsection{Problem 1}\label{problem-1}}

Let \(Y_1,...,Y_n\) be a random sample where
\(Y_i \sim \text{GAM}(\theta, \kappa)\).

\begin{itemize}
\tightlist
\item
  \(Y_i\) represents the amount of rainfall accumulated in a reservoir
  over a year
\item
  \(\mu = E(YPi) = 132\)
\item
  \(\sigma^2 = Var(Y_i) = 26.4\) \(\implies \sigma = 5.14\)
\item
  \(Y_n = 1/n \sum_{i = 1}^{n} X_i\)

  \begin{itemize}
  \tightlist
  \item
    Note: I'm not sure if this was to be
    \(Y_n = 1/n \sum_{i = 1}^{n} Y_i\)
  \end{itemize}
\end{itemize}

\textbf{Part (a):} Give an approximation of \(P(\bar{Y} > 200)\) for
\(n = 500\)

\begin{quote}
To calculate this probability, we're going to rely on the central limit
theorem. We will let
\(Z = \frac{\bar{Y}_n - \mu}{\sigma/\sqrt{n}} = \frac{\bar{Y}_n - 132}{5.14/\sqrt{500}}\).
\end{quote}

\begin{align*}
P(\bar{Y}_n \ge 200) &= P(\frac{\bar{Y}_n - 132}{0.23} \ge \frac{200 - 132}{0.23})\\
&= 1 - P(\frac{\bar{Y}_n - 132}{0.23} < \frac{200 - 132}{0.23})\\
&= 1 - P(Z < 295.9)\\
&= 0
\end{align*}

\textbf{Part (b):} What is the true distribution of \(\bar{Y}_n\)?

\begin{quote}
I decided to use the MGF method to deterimine this distribution:
\end{quote}

\begin{align*}
M_{Y_i}(t) &= E(exp(Y_it))\\
&= (\frac{1}{1 - \theta t})^{\kappa}\\
M_{\bar{Y}_n}(t) &= E(exp(\bar{Y}_n t))\\
&= E(exp((1/n)\sum Y_i t))\\
&= E(exp((t/n)\sum Y_i))\\
&= E(\prod exp((Y_i t)/n))\\
&= \prod E(exp((Y_i t)/n)))\\
&= \prod M_{Y_i}(t/n)\\
&= (\frac{1}{1 - \theta t/n})^{n\kappa}
\end{align*}

\begin{quote}
You can see that \(\bar{Y}_n\) follows a Gamma distribution with
parameters \(GAM(\theta/n, n\kappa)\).
\end{quote}

\textbf{Part (c):} Does \(\bar{Y}_n\) converge in probability? If so,
what is the limit?

\begin{quote}
Let's start with the definition of \emph{stochastic convergence}. That
is, \(Y_N\) is said to converge in probability (stochastically) to a
constant c if and only if for every \(\epsilon > 0\),
\(\lim_{n \to \infty} P[|Y_n- c| < \epsilon] = 1\) (see chapter 7 notes,
page 17).

We can also remember the Chebychev inequality to help us determine
stochastic convergence. Remember that probabilities are bounded by 0 and
1, which when combined with the Chebychev inequality, gives rise to the
following:
\end{quote}

\hypertarget{problem-2}{%
\subsection{Problem 2}\label{problem-2}}

Suppose that \(X_n\) follows a binomial distribution with parameters
\((n, \pi)\) where \(\pi \in (0,1)\). Define \(Y_n = \frac{1}{n}X_n\)
for each \(n\).

In each case below, clearly indicate which theorems you have used to
establish the result.

\textbf{Hint:} First, write \textbf{Y\_n} as a sample mean of a random
sample that you will specify.

\begin{quote}
We can specify \(Y_n\) as a sample mean by considering that
\(X_n = \sum_{i = 1}^{n} W_i, W_i \sim Binom(1, \pi)\). This will help
us in the following parts.
\end{quote}

\textbf{Part (a):} Find the asymptotic normal distribution of \(Y_n\).

\begin{quote}
By the central limit theorem and definition 7.5.1, if it can be shown
that \(\frac{Y_n - m}{c/\sqrt{n}} \sim N(0,1)\) as
\(n \rightarrow \infty\), then the asymptotic normal distribution of
\(Y_n\) is \(Y_n \sim^d N(m, c^2/n)\).

Because we're dealing with the sample mean of \(n\) \(i.i.d.\) random
variables of \(W_i\), where \(W_i \sim Binom(1,\pi)\), we know that the
asymptotic normal distribution of \(Y_n\) is \$N(\pi,
\frac{\pi(1 - \pi)}{n}).
\end{quote}

\textbf{Part (b):} Find the asymptotic normal distribution of
\(\ln(Y_n)\).

\hypertarget{problem-3}{%
\subsection{Problem 3}\label{problem-3}}

Let \(W_i\) be the weight of the \(i\)th airline passenger's luggage.
Assume that the weights are independent, each with pdf
\(f(w) = \theta B^{-\theta} w^{\theta-1}, 0 < w < B, \quad 0 \quad o.w.\)

\textbf{Part (a):} For \(n = 100, \theta = 3, B = 80\), approximate
\(P\left[\sum_{i = 1}^{100} W_i > 6025\right]\)

\begin{quote}
By the CLT, we know that \(\sum W_i\) is approximately distributed
normally with parameters N(n\mu, n \sigma\^{}2). Using an online
integral calculator, I calculated \(E(W_i)\) to be 60 and \(Var(W_i)\)
to be 240. Thus, this sum is approximated as follows: \$\sum\_\{i =
1\}\^{}\{n\}
\end{quote}

\textbf{Part (b):} If W\_\{1:n\} is the smallest value of n,

\hypertarget{problem-4}{%
\subsection{Problem 4}\label{problem-4}}

\end{document}
