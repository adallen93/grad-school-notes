% Options for packages loaded elsewhere
\PassOptionsToPackage{unicode}{hyperref}
\PassOptionsToPackage{hyphens}{url}
%
\documentclass[
]{article}
\usepackage{amsmath,amssymb}
\usepackage{iftex}
\ifPDFTeX
  \usepackage[T1]{fontenc}
  \usepackage[utf8]{inputenc}
  \usepackage{textcomp} % provide euro and other symbols
\else % if luatex or xetex
  \usepackage{unicode-math} % this also loads fontspec
  \defaultfontfeatures{Scale=MatchLowercase}
  \defaultfontfeatures[\rmfamily]{Ligatures=TeX,Scale=1}
\fi
\usepackage{lmodern}
\ifPDFTeX\else
  % xetex/luatex font selection
\fi
% Use upquote if available, for straight quotes in verbatim environments
\IfFileExists{upquote.sty}{\usepackage{upquote}}{}
\IfFileExists{microtype.sty}{% use microtype if available
  \usepackage[]{microtype}
  \UseMicrotypeSet[protrusion]{basicmath} % disable protrusion for tt fonts
}{}
\makeatletter
\@ifundefined{KOMAClassName}{% if non-KOMA class
  \IfFileExists{parskip.sty}{%
    \usepackage{parskip}
  }{% else
    \setlength{\parindent}{0pt}
    \setlength{\parskip}{6pt plus 2pt minus 1pt}}
}{% if KOMA class
  \KOMAoptions{parskip=half}}
\makeatother
\usepackage{xcolor}
\usepackage[margin=1in]{geometry}
\usepackage{graphicx}
\makeatletter
\def\maxwidth{\ifdim\Gin@nat@width>\linewidth\linewidth\else\Gin@nat@width\fi}
\def\maxheight{\ifdim\Gin@nat@height>\textheight\textheight\else\Gin@nat@height\fi}
\makeatother
% Scale images if necessary, so that they will not overflow the page
% margins by default, and it is still possible to overwrite the defaults
% using explicit options in \includegraphics[width, height, ...]{}
\setkeys{Gin}{width=\maxwidth,height=\maxheight,keepaspectratio}
% Set default figure placement to htbp
\makeatletter
\def\fps@figure{htbp}
\makeatother
\setlength{\emergencystretch}{3em} % prevent overfull lines
\providecommand{\tightlist}{%
  \setlength{\itemsep}{0pt}\setlength{\parskip}{0pt}}
\setcounter{secnumdepth}{-\maxdimen} % remove section numbering
\ifLuaTeX
  \usepackage{selnolig}  % disable illegal ligatures
\fi
\IfFileExists{bookmark.sty}{\usepackage{bookmark}}{\usepackage{hyperref}}
\IfFileExists{xurl.sty}{\usepackage{xurl}}{} % add URL line breaks if available
\urlstyle{same}
\hypersetup{
  pdftitle={BIOSTAT 704 - Homework 1},
  pdfauthor={Austin Allen},
  hidelinks,
  pdfcreator={LaTeX via pandoc}}

\title{BIOSTAT 704 - Homework 1}
\author{Austin Allen}
\date{January 30, 2024}

\begin{document}
\maketitle

\hypertarget{problem-1}{%
\subsubsection{Problem 1}\label{problem-1}}

\textbf{BE Exercise 7.1: Consider a random sample of size n from a
distribution with CDF \(F(x) = 1 - 1/x\) if \(1 \le x < \infty\), and
zero otherwise.}

\hypertarget{a-derive-the-cdf-of-the-smallest-order-statistic-x_1n}{%
\paragraph{\texorpdfstring{a) Derive the CDF of the smallest order
statistic,
\(X_{1:n}\)}{a) Derive the CDF of the smallest order statistic, X\_\{1:n\}}}\label{a-derive-the-cdf-of-the-smallest-order-statistic-x_1n}}

\begin{align*}
G_n(y) &= P(Y_n \le y)\\
&= 1 - P(Y_n > y)\\
&= 1 - P(X_{1:n} > y)\\
&= 1 - P(X_1 > y, X_2 > y,...,X_n > y)\\
&= 1 - P(X_1 > y)P(X_2 > y)...P(X_n > y)\\
&= 1 - [P(X_i > y)]^n\\
&= 1 - [1 - P(X_i \le y)]^n\\
&= 1 - [1 - F_{X}(x)]^n\\
&= 1 - [1 - (1 - 1/x)]^n\\
&= 1 - (1/x)^n\\
G_{X_{1:n}}(x) &= 
\begin{cases}
  1 - \frac{1}{x^n}, &\text{ for } x \ge 1\\
  0, &\text{ o.w.}
\end{cases}
\end{align*}

\hypertarget{b-find-the-limiting-distribution-of-x_1n}{%
\paragraph{\texorpdfstring{b) Find the limiting distribution of
\(X_{1:n}\)}{b) Find the limiting distribution of X\_\{1:n\}}}\label{b-find-the-limiting-distribution-of-x_1n}}

For \(x \ge 1\),

\begin{align*}
\lim_{n \to \infty} 1 - 1/x^n &= 1
\end{align*} Thus the limiting distribution of G\_\{X\_\{1:n\}\}(x) is,
\begin{align*}
G(y) &= 
\begin{cases}
  1, &\text{ for } x \ge 1\\
  0, &\text{ o.w.}
\end{cases}
\end{align*}

\hypertarget{c-find-the-limiting-distribution-of-x_1nn}{%
\paragraph{\texorpdfstring{c) Find the limiting distribution of
\(X_{1:n}^n\)}{c) Find the limiting distribution of X\_\{1:n\}\^{}n}}\label{c-find-the-limiting-distribution-of-x_1nn}}

\textbf{BE Exercise 7.18}

\hypertarget{problem-2}{%
\subsubsection{Problem 2}\label{problem-2}}

\textbf{BE Exercise 7.2}

\textbf{BE Exercise 7.19}

\textbf{BE Exercise 7.3}

\hypertarget{problem-3}{%
\subsubsection{Problem 3}\label{problem-3}}

\hypertarget{consider-example-7.2.2-in-the-textbook-on-pages-233.-let-x1-x2-.-.-.-x_n-be-a-random-sample-from-an-exponential-distribution-xi-expux3b8-and-let-y_n-x_1n-be-the-smallest-order-statistic.}{%
\paragraph{\texorpdfstring{Consider Example 7.2.2 in the textbook, on
pages 233. Let \(X1, X2, . . . , X_n\) be a random sample from an
exponential distribution, Xi ∼ EXP(θ) and let \(Y_n = X_{1:n}\) be the
smallest order
statistic.}{Consider Example 7.2.2 in the textbook, on pages 233. Let X1, X2, . . . , X\_n be a random sample from an exponential distribution, Xi ∼ EXP(θ) and let Y\_n = X\_\{1:n\} be the smallest order statistic.}}\label{consider-example-7.2.2-in-the-textbook-on-pages-233.-let-x1-x2-.-.-.-x_n-be-a-random-sample-from-an-exponential-distribution-xi-expux3b8-and-let-y_n-x_1n-be-the-smallest-order-statistic.}}

\hypertarget{a-using-the-definition-of-a-cdf-theorem-2.2.3-show-that-g_ny-is-a-proper-cdf.-that-is-show-that-the-conditions-2.2.82.2.11-hold-for-g_ny.}{%
\subparagraph{\texorpdfstring{a) Using the definition of a CDF (Theorem
2.2.3), show that \(G_n(Y)\) is a proper CDF. That is, show that the
conditions (2.2.8--2.2.11) hold for
\(G_n(y)\).}{a) Using the definition of a CDF (Theorem 2.2.3), show that G\_n(Y) is a proper CDF. That is, show that the conditions (2.2.8--2.2.11) hold for G\_n(y).}}\label{a-using-the-definition-of-a-cdf-theorem-2.2.3-show-that-g_ny-is-a-proper-cdf.-that-is-show-that-the-conditions-2.2.82.2.11-hold-for-g_ny.}}

\hypertarget{b-sketch-the-graph-of-the-limiting-function-gy.}{%
\subparagraph{\texorpdfstring{b) Sketch the graph of the limiting
function
\(G(Y)\).}{b) Sketch the graph of the limiting function G(Y).}}\label{b-sketch-the-graph-of-the-limiting-function-gy.}}

\hypertarget{c-explain-why-the-authors-state-that-gy-being-discontinuous-and-not-even-right-continuous-at-y-0-is-not-a-problem.-how-does-this-connect-with-our-explanations-in-the-lecture-notes-on-pages-12}{%
\subparagraph{c) Explain why the authors state that G(y) being
discontinuous (and not even right-continuous) at y = 0 ''is not a
problem''. How does this connect with our explanations in the lecture
notes on pages
1--2?}\label{c-explain-why-the-authors-state-that-gy-being-discontinuous-and-not-even-right-continuous-at-y-0-is-not-a-problem.-how-does-this-connect-with-our-explanations-in-the-lecture-notes-on-pages-12}}

\hypertarget{problem-4}{%
\subsubsection{Problem 4}\label{problem-4}}

\hypertarget{example-7.2.6-in-the-textbook-on-page-235}{%
\paragraph{(Example 7.2.6 in the textbook, on page
235)}\label{example-7.2.6-in-the-textbook-on-page-235}}

\hypertarget{consider-x1-.-.-.-x_n-a-random-sample-where-x_i-expux3b8.-define-the-random-sequence-y_n-1ux3b8x_nn-lnn.-the-purpose-of-this-exercise-is-to-demonstrate-that-y_n-converges-in-distribution}{%
\paragraph{\texorpdfstring{Consider \(X1, . . . , X_n\) a random sample
where \(X_i ∼ EXP(θ)\). Define the random sequence
\(Y_n = (1/θ)X_{n:n} − \ln(n)\). The purpose of this exercise is to
demonstrate that \(Y_n\) converges in
distribution}{Consider X1, . . . , X\_n a random sample where X\_i ∼ EXP(θ). Define the random sequence Y\_n = (1/θ)X\_\{n:n\} − \textbackslash ln(n). The purpose of this exercise is to demonstrate that Y\_n converges in distribution}}\label{consider-x1-.-.-.-x_n-a-random-sample-where-x_i-expux3b8.-define-the-random-sequence-y_n-1ux3b8x_nn-lnn.-the-purpose-of-this-exercise-is-to-demonstrate-that-y_n-converges-in-distribution}}

\hypertarget{a-prove-that-the-cdf-of-y_n-is}{%
\subparagraph{\texorpdfstring{a) Prove that the CDF of \(Y_n\)
is}{a) Prove that the CDF of Y\_n is}}\label{a-prove-that-the-cdf-of-y_n-is}}

\begin{align*}
G_n(Y) &=
\begin{cases}
  \left[ 1 - \frac{1}{n} e^{-y} \right]^n, & \text{when } y > \ln(n) \\
  0, & \text{otherwise}
\end{cases}
\end{align*}

\hypertarget{b-show-that-lim_n-to-infty-g_ny-gy-with-e-e-y-where---infty-y-infty.}{%
\subparagraph{\texorpdfstring{b) Show that
\(\lim_{n \to \infty} G_n(Y) = G(Y)\), with \(e^{-e^{-y}}\), where
\(- \infty < y < \infty\).}{b) Show that \textbackslash lim\_\{n \textbackslash to \textbackslash infty\} G\_n(Y) = G(Y), with e\^{}\{-e\^{}\{-y\}\}, where - \textbackslash infty \textless{} y \textless{} \textbackslash infty.}}\label{b-show-that-lim_n-to-infty-g_ny-gy-with-e-e-y-where---infty-y-infty.}}

\hypertarget{c-show-that-gy-is-a-cdf.}{%
\subparagraph{\texorpdfstring{c) Show that \(G(Y)\) is a
CDF.}{c) Show that G(Y) is a CDF.}}\label{c-show-that-gy-is-a-cdf.}}

\hypertarget{problem-5}{%
\subsubsection{Problem 5}\label{problem-5}}

\textbf{BE Exercise 7.5}

\end{document}
