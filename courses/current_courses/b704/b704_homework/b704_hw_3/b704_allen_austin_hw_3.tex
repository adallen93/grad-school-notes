% Options for packages loaded elsewhere
\PassOptionsToPackage{unicode}{hyperref}
\PassOptionsToPackage{hyphens}{url}
%
\documentclass[
]{article}
\usepackage{amsmath,amssymb}
\usepackage{iftex}
\ifPDFTeX
  \usepackage[T1]{fontenc}
  \usepackage[utf8]{inputenc}
  \usepackage{textcomp} % provide euro and other symbols
\else % if luatex or xetex
  \usepackage{unicode-math} % this also loads fontspec
  \defaultfontfeatures{Scale=MatchLowercase}
  \defaultfontfeatures[\rmfamily]{Ligatures=TeX,Scale=1}
\fi
\usepackage{lmodern}
\ifPDFTeX\else
  % xetex/luatex font selection
\fi
% Use upquote if available, for straight quotes in verbatim environments
\IfFileExists{upquote.sty}{\usepackage{upquote}}{}
\IfFileExists{microtype.sty}{% use microtype if available
  \usepackage[]{microtype}
  \UseMicrotypeSet[protrusion]{basicmath} % disable protrusion for tt fonts
}{}
\makeatletter
\@ifundefined{KOMAClassName}{% if non-KOMA class
  \IfFileExists{parskip.sty}{%
    \usepackage{parskip}
  }{% else
    \setlength{\parindent}{0pt}
    \setlength{\parskip}{6pt plus 2pt minus 1pt}}
}{% if KOMA class
  \KOMAoptions{parskip=half}}
\makeatother
\usepackage{xcolor}
\usepackage[margin=1in]{geometry}
\usepackage{color}
\usepackage{fancyvrb}
\newcommand{\VerbBar}{|}
\newcommand{\VERB}{\Verb[commandchars=\\\{\}]}
\DefineVerbatimEnvironment{Highlighting}{Verbatim}{commandchars=\\\{\}}
% Add ',fontsize=\small' for more characters per line
\usepackage{framed}
\definecolor{shadecolor}{RGB}{248,248,248}
\newenvironment{Shaded}{\begin{snugshade}}{\end{snugshade}}
\newcommand{\AlertTok}[1]{\textcolor[rgb]{0.94,0.16,0.16}{#1}}
\newcommand{\AnnotationTok}[1]{\textcolor[rgb]{0.56,0.35,0.01}{\textbf{\textit{#1}}}}
\newcommand{\AttributeTok}[1]{\textcolor[rgb]{0.13,0.29,0.53}{#1}}
\newcommand{\BaseNTok}[1]{\textcolor[rgb]{0.00,0.00,0.81}{#1}}
\newcommand{\BuiltInTok}[1]{#1}
\newcommand{\CharTok}[1]{\textcolor[rgb]{0.31,0.60,0.02}{#1}}
\newcommand{\CommentTok}[1]{\textcolor[rgb]{0.56,0.35,0.01}{\textit{#1}}}
\newcommand{\CommentVarTok}[1]{\textcolor[rgb]{0.56,0.35,0.01}{\textbf{\textit{#1}}}}
\newcommand{\ConstantTok}[1]{\textcolor[rgb]{0.56,0.35,0.01}{#1}}
\newcommand{\ControlFlowTok}[1]{\textcolor[rgb]{0.13,0.29,0.53}{\textbf{#1}}}
\newcommand{\DataTypeTok}[1]{\textcolor[rgb]{0.13,0.29,0.53}{#1}}
\newcommand{\DecValTok}[1]{\textcolor[rgb]{0.00,0.00,0.81}{#1}}
\newcommand{\DocumentationTok}[1]{\textcolor[rgb]{0.56,0.35,0.01}{\textbf{\textit{#1}}}}
\newcommand{\ErrorTok}[1]{\textcolor[rgb]{0.64,0.00,0.00}{\textbf{#1}}}
\newcommand{\ExtensionTok}[1]{#1}
\newcommand{\FloatTok}[1]{\textcolor[rgb]{0.00,0.00,0.81}{#1}}
\newcommand{\FunctionTok}[1]{\textcolor[rgb]{0.13,0.29,0.53}{\textbf{#1}}}
\newcommand{\ImportTok}[1]{#1}
\newcommand{\InformationTok}[1]{\textcolor[rgb]{0.56,0.35,0.01}{\textbf{\textit{#1}}}}
\newcommand{\KeywordTok}[1]{\textcolor[rgb]{0.13,0.29,0.53}{\textbf{#1}}}
\newcommand{\NormalTok}[1]{#1}
\newcommand{\OperatorTok}[1]{\textcolor[rgb]{0.81,0.36,0.00}{\textbf{#1}}}
\newcommand{\OtherTok}[1]{\textcolor[rgb]{0.56,0.35,0.01}{#1}}
\newcommand{\PreprocessorTok}[1]{\textcolor[rgb]{0.56,0.35,0.01}{\textit{#1}}}
\newcommand{\RegionMarkerTok}[1]{#1}
\newcommand{\SpecialCharTok}[1]{\textcolor[rgb]{0.81,0.36,0.00}{\textbf{#1}}}
\newcommand{\SpecialStringTok}[1]{\textcolor[rgb]{0.31,0.60,0.02}{#1}}
\newcommand{\StringTok}[1]{\textcolor[rgb]{0.31,0.60,0.02}{#1}}
\newcommand{\VariableTok}[1]{\textcolor[rgb]{0.00,0.00,0.00}{#1}}
\newcommand{\VerbatimStringTok}[1]{\textcolor[rgb]{0.31,0.60,0.02}{#1}}
\newcommand{\WarningTok}[1]{\textcolor[rgb]{0.56,0.35,0.01}{\textbf{\textit{#1}}}}
\usepackage{graphicx}
\makeatletter
\def\maxwidth{\ifdim\Gin@nat@width>\linewidth\linewidth\else\Gin@nat@width\fi}
\def\maxheight{\ifdim\Gin@nat@height>\textheight\textheight\else\Gin@nat@height\fi}
\makeatother
% Scale images if necessary, so that they will not overflow the page
% margins by default, and it is still possible to overwrite the defaults
% using explicit options in \includegraphics[width, height, ...]{}
\setkeys{Gin}{width=\maxwidth,height=\maxheight,keepaspectratio}
% Set default figure placement to htbp
\makeatletter
\def\fps@figure{htbp}
\makeatother
\setlength{\emergencystretch}{3em} % prevent overfull lines
\providecommand{\tightlist}{%
  \setlength{\itemsep}{0pt}\setlength{\parskip}{0pt}}
\setcounter{secnumdepth}{-\maxdimen} % remove section numbering
\ifLuaTeX
  \usepackage{selnolig}  % disable illegal ligatures
\fi
\IfFileExists{bookmark.sty}{\usepackage{bookmark}}{\usepackage{hyperref}}
\IfFileExists{xurl.sty}{\usepackage{xurl}}{} % add URL line breaks if available
\urlstyle{same}
\hypersetup{
  pdftitle={BIOSTAT 704 - Homework 3},
  pdfauthor={Austin Allen},
  hidelinks,
  pdfcreator={LaTeX via pandoc}}

\title{BIOSTAT 704 - Homework 3}
\author{Austin Allen}
\date{February 15th, 2024}

\begin{document}
\maketitle

\hypertarget{problem-1-be-exercise-8.15}{%
\subsection{Problem 1: BE Exercise
8.15}\label{problem-1-be-exercise-8.15}}

For BE 8.15, you must justify each result using a theorem(s) and/or
definition(s); that is, you will NOT get full credit for just listing
the distribution.

Supose that \(X_i \sim N(\mu, \sigma^2), i = 1,...,n\) and
\(Z_i \sim N(0,1), i = 1,...,k\), and all variables are independent.
State the distribution of each of the following variables if it is a
``names'' distribution or otherwise state ``unknown.''

\textbf{Part (a):} \(X_1 - X_2\)

\begin{quote}
Theorem 8.3.1 describes the linear combination of independent normal
random variables.

Let \(Y = X_1 - X_2\).
\end{quote}

\begin{align*}
&Y \sim N\left(\sum_{i = 1}^{n} a_i \mu_i, \sum_{i = 1}^{n} a_i^2 \sigma_i^2\right)\\
&\implies Y \sim N\left((1) \mu + (-1)\mu, (1)^2\sigma^2 + (-1)^2 \sigma^2\right)\\
&\implies Y \sim N(0, 2\sigma^2)
\end{align*}

\textbf{Part (b):} \(X_2 + 2X_3\)

\begin{quote}
By applying the same results from theorem 8.3.1, we can see that for
\(Y = X_2 + 2X_3\):
\end{quote}

\begin{align*}
&Y \sim N\left(\sum_{i = 1}^{n} a_i \mu_i, \sum_{i = 1}^{n} a_i^2 \sigma_i^2\right)\\
&\implies Y \sim N\left((1) \mu + (2)\mu, (1)^2\sigma^2 + (2)^2 \sigma^2\right)\\
&\implies Y \sim N(3\mu, 5\sigma^2)
\end{align*}

\textbf{Part (c):} \(\frac{X_1 - X_2}{\sigma S_Z \sqrt{2}}\)

\begin{quote}
\((X_1 - X_2)/\sqrt{2}\sigma \sim N(0,1)\). Thus, by theorem 8.4.2,
\(\frac{X_1 - X_2}{\sigma S_Z \sqrt{2}} \sim t(k - 1)\) (note:
\(\mu = 0\)).
\end{quote}

\textbf{Part (d):} \(Z_1^2\)

\begin{quote}
We know from theorem 8.3.5 that if \(Z\sim N(0,1)\), then
\(Z^2 \sim \chi^2(1)\). Thus, \(Z_1^2 \sim \chi^2(1)\).
\end{quote}

\textbf{Part (e):} \(\frac{\sqrt{n} (\bar{X} - \mu)}{\sigma S_Z}\)

\begin{quote}
Similarly to part c), \((\bar{X} - \mu)/(\sigma/\sqrt{n}) \sim N(0,1)\).
Thus, \(\frac{\sqrt{n} (\bar{X} - \mu)}{\sigma S_Z} \sim t(k - 1)\)
\end{quote}

\textbf{Part (f):} \(Z_1^2 + Z_2^2\)

\begin{quote}
We know that \(Z^2 \sim \chi^2(1)\) (see theorem 8.3.5). We can combine
this with the fact that the sum of independent \(\chi^2\) random
variables follows a \(\chi^2\) distribution with
\(\sum_{i = 1}^{n} \nu_i\) degrees of freedom. Thus, for
\(Y = Z_1^2 + Z_2^2, Y \sim \chi^2(2)\).
\end{quote}

\textbf{Part (g):} \(Z_1^2 - Z_2^2\)

\begin{quote}
Unfortunately, the exact distribution of \(Z_1^2 - Z_2^2\) is not
possible to derive with the theorems we've learned to this point.
\end{quote}

\textbf{Part (h):} \(\frac{Z_1}{\sqrt{Z_2^2}}\)

\begin{quote}
From theorem 8.4.1, we know that for independent random variables \(V\)
and \(Z\), \(T = \frac{Z}{\sqrt{V/\nu}} \sim t(\nu)\). We also know and
have used theorem 8.3.5 which shows that \(Z^2 \sim \chi^2(1)\). Thus,
for \(T = \frac{Z_1}{\sqrt{Z_2^2}}, T \sim t(1)\).
\end{quote}

\textbf{Part (i):} \(\frac{Z_1^2}{Z_2^2}\)

\begin{quote}
Theorem 8.4.4 describes a random variable
\(X = \frac{V_1/\nu_1}{V_2/\nu_2}, X \sim F(\nu_1, \nu_2)\), where
\(V_1 \sim \chi^2(\nu_1)\) and \(V_2 \sim \chi^2(\nu_2)\). Combining
this with the fact that \(Z^2 \sim \chi^2(1)\), we can deduce that for
\(X = \frac{Z_1^2}{Z_2^2}, X \sim F(1, 1)\).
\end{quote}

\textbf{Part (k):} \(\frac{\bar{X}}{\bar{Z}}\)

\begin{quote}
Unfortunately, we do not have enough information to derive the exact
distribution of \(\frac{\bar{X}}{\bar{Z}}\).
\end{quote}

\textbf{Part (l):}
\(\frac{\sqrt{nk}(\bar{X} - \mu)}{\sigma \sqrt{\sum_{i = 1}^k} Z_i^2}\)

\begin{quote}
We can rewrite this to be in this form:
\end{quote}

\[
\frac{\frac{(\bar{X} - \mu)}{\sigma/\sqrt{n}}}{\frac{ \sqrt{\sum_{i = 1}^k} Z_i^2}{\sqrt{k}}}
\]

\begin{quote}
The denominator follows a standard normal distribution and
\(\sum_{i = 1}^{k} Z_i^2\) follows a \(\chi^2\) distribution with \(k\)
degrees of freedom. Thus, applying theorem 8.4.1, for
\(Y = \frac{\sqrt{nk}(\bar{X} - \mu)}{\sigma \sqrt{\sum_{i = 1}^k} Z_i^2}, Y \sim t(k)\).
\end{quote}

\textbf{Part (m):}
\(\frac{\sum_{i = 1}^{n}(X_i - \mu)^2}{\sigma^2} + \sum_{i = 1}^{k}(Z_i - \bar{Z})^2\)

\begin{quote}
We can recognize \(\frac{\sum_{i = 1}^{n}(X_i - \mu)^2}{\sigma^2}\) as
the sample variance, which we know follows a \(\chi^2\) distribution
with \(n - 1\) degrees of freedom. We also know that
\(\sum_{i = 1}^{k} (Z_i - \bar{Z})^2\) follows a \(\chi^2\) distribution
with \(k\) degrees of freedom. Lastly, we know that the sum of two
\(\chi^2\) random variables with degrees of freedom \(\nu_1\) and
\(\nu_2\) results in a \(\chi^2\) random variable with \(\nu_1 + \nu_2\)
degrees of freedom. Thus, for
\(Y = \frac{\sum_{i = 1}^{n}(X_i - \mu)^2}{\sigma^2} + \sum_{i = 1}^{k}(Z_i - \bar{Z})^2, Y \sim \chi^2(k + n - 1)\).
\end{quote}

\textbf{Part (n):}
\(\frac{\bar{X}}{\sigma^2} + \frac{\sum_{i = 1}^{k} Z_i}{k}\)

\begin{quote}
We know that
\(\frac{\bar{X}}{\sigma^2} \sim N(\mu/\sigma^2, 1/\sigma^2n)\) (sum of
independent Normal random variables). We also know that
\(\frac{\sum_{i = 1}^{k} Z_i}{k} \sim N(k\mu/k, k/k^2) \implies \frac{\sum_{i = 1}^{k} Z_i}{k} \sim N(0, 1/k)\)
(also by the sum of normal independent random variables). Thus, by
adding these two expression together, we get
\(Y = \frac{\bar{X}}{\sigma^2} + \frac{\sum_{i = 1}^{k} Z_i}{k}, Y \sim N(\mu/\sigma^2, 1/n\sigma^2 + 1/k)\).
\end{quote}

\textbf{Part (o):} \(k\bar{Z^2}\)

\begin{quote}
\(k\bar{Z}^2 = (k)(1/k)\sum_{i = 1}^{k} Z_i^2 \sim \chi^2(k)\) (sum of
independent \(\chi^2\) random variables).
\end{quote}

\textbf{Part (p):}
\(\frac{(k - 1) \sum_{i = 1}^{n} (X_i - \bar{X})^2}{(n - 1)\sigma^2 \sum_{i = 1}^{k} (Z_i - \bar{Z})^2}\)

\begin{quote}
Note that the numerator is the sample variance for \(X_i\) which is
distributed as a \(\chi^2\) random variable with \(n - 1\) degrees of
freedom. Similarly, the denominator is the sample variance of \(Z_i\)
and is distributed as a \(\chi^2\) random variable with \(k - 1\)
degrees of freedom. Thus, by theorem 8.4.4,
\(\frac{(k - 1) \sum_{i = 1}^{n} (X_i - \bar{X})^2}{(n - 1)\sigma^2 \sum_{i = 1}^{k} (Z_i - \bar{Z})^2} \sim F(n - 1, k - 1)\).
\end{quote}

\hypertarget{problem-2-let-x_i-for-i-1-2-3-be-independent-random-variables-with-ni-i2-distributions.}{%
\subsection{\texorpdfstring{Problem 2: Let \(X_i\) for \(i = 1, 2, 3\)
be independent random variables with \(N(i, i^2)\)
distributions.}{Problem 2: Let X\_i for i = 1, 2, 3 be independent random variables with N(i, i\^{}2) distributions.}}\label{problem-2-let-x_i-for-i-1-2-3-be-independent-random-variables-with-ni-i2-distributions.}}

For each of the following situations, use the \(X_i\)'s to construct a
statistic with the indicated distribution. That is, your solution to
each part below should be a function of \textbf{ALL THREE} \(X_i\)'s.

\begin{quote}
For reference, I will explicitly state the distribution of all three
\(X_i\)'s:

\begin{itemize}
\tightlist
\item
  \(X_1 \sim N(1,1)\)
\item
  \(X_2 \sim N(2, 4)\)
\item
  \(X_3 \sim N(3, 9)\)
\end{itemize}
\end{quote}

\textbf{Part (a):} Chi-square distribution with 3 degrees of freedom.

\begin{quote}
\((X_1 - 1)^2 + \frac{(X_2 - 2)^2}{4} + \frac{(X_3 - 3)^2}{9} \sim \chi^2(3)\)
\end{quote}

\textbf{Part (b):} Student t distribution with 2 degrees of freedom.

\begin{quote}
\(\frac{\frac{X_3 - 3}{3}}{\sqrt{\left[(X_1 - 1)^2 + \frac{(X_2 - 2)^2}{4}\right]/2} } \sim t(2)\)
\end{quote}

\textbf{Part (c):} F distribution with 1 and 2 degrees of freedom.

\begin{quote}
\(\frac{\frac{(X_3 - 3)^2}{9}}{\left[(X_1 - 1)^2 + \frac{(X_2 - 2)^2}{4}\right]/2 } \sim F(1,2)\)
\end{quote}

\hypertarget{problem-3-be-exercises-8.13-8.18.}{%
\subsection{Problem 3: BE Exercises: 8.13,
8.18.}\label{problem-3-be-exercises-8.13-8.18.}}

For the BE exercises listed above, you may use R or statistical tables
to find the probabilities. However, you are encouraged to do it both
ways for practice.

\hypertarget{consider-independent-random-variables-z_i-sim-n01-i-1...16-and-let-barz-be-the-sample-mean.}{%
\subsubsection{\texorpdfstring{8.13: consider independent random
variables \(Z_i \sim N(0,1), i = 1,...,16\), and let \(\bar{Z}\) be the
sample
mean.}{8.13: consider independent random variables Z\_i \textbackslash sim N(0,1), i = 1,...,16, and let \textbackslash bar\{Z\} be the sample mean.}}\label{consider-independent-random-variables-z_i-sim-n01-i-1...16-and-let-barz-be-the-sample-mean.}}

Find the following:

\textbf{Part (a):} \(P[\bar{Z} < 1/2]\)

\begin{quote}
For \(Y = \bar{Z}, Y \sim N(0, 1/16)\)
\end{quote}

\begin{Shaded}
\begin{Highlighting}[]
\CommentTok{\# Set variables}
\NormalTok{mu }\OtherTok{\textless{}{-}} \DecValTok{0}
\NormalTok{standard\_deviation }\OtherTok{\textless{}{-}} \FunctionTok{sqrt}\NormalTok{(}\DecValTok{1}\SpecialCharTok{/}\DecValTok{16}\NormalTok{)}
\NormalTok{q }\OtherTok{\textless{}{-}} \DecValTok{1}\SpecialCharTok{/}\DecValTok{2}
\CommentTok{\# Calculate probability}
\NormalTok{prob }\OtherTok{\textless{}{-}} \FunctionTok{pnorm}\NormalTok{(q, mu, standard\_deviation)}
\CommentTok{\# Print probability}
\FunctionTok{print}\NormalTok{(}\FunctionTok{paste}\NormalTok{(}\StringTok{"The probability that the sample mean is less than 0.5 is"}\NormalTok{, }\FunctionTok{round}\NormalTok{(prob, }\DecValTok{3}\NormalTok{)))}
\end{Highlighting}
\end{Shaded}

\begin{verbatim}
## [1] "The probability that the sample mean is less than 0.5 is 0.977"
\end{verbatim}

\textbf{Part (b):} \(P[Z_1 - Z_2 < 2]\)

\begin{quote}
For \(Y = Z_1 - Z_2, Y \sim N(0, 2)\).
\end{quote}

\begin{Shaded}
\begin{Highlighting}[]
\CommentTok{\# Set variables}
\NormalTok{mu }\OtherTok{\textless{}{-}} \DecValTok{0}
\NormalTok{standard\_deviation }\OtherTok{\textless{}{-}} \FunctionTok{sqrt}\NormalTok{(}\DecValTok{2}\NormalTok{)}
\NormalTok{q }\OtherTok{\textless{}{-}} \DecValTok{2}
\CommentTok{\# Calculate probability}
\NormalTok{prob }\OtherTok{\textless{}{-}} \FunctionTok{pnorm}\NormalTok{(q, mu, standard\_deviation)}
\CommentTok{\# Print probability}
\FunctionTok{print}\NormalTok{(}\FunctionTok{paste}\NormalTok{(}\StringTok{"The probability that the sample mean is less than 0.5 is"}\NormalTok{, }\FunctionTok{round}\NormalTok{(prob, }\DecValTok{3}\NormalTok{)))}
\end{Highlighting}
\end{Shaded}

\begin{verbatim}
## [1] "The probability that the sample mean is less than 0.5 is 0.921"
\end{verbatim}

\textbf{Part (c):} \(P[Z_1 + Z_2 < 2]\)

\begin{quote}
For \(Y = Z_1 + Z_2, Y \sim N(0, 2)\). Note that we should get to the
same probability that we got in part (b). However, this time I'm going
to do it with a table (just to shake things up).

I first have to standardize \(Y\) by subtracting the mean (0) and
dividing by the standard deviation (\(\sqrt{2}\)). This will help me
search for the probability in the standard normal CDF table.

The quantile I'm interested in is \(Z = 2/\sqrt{2} \approx 1.41\). Sure
enough, at this spot in the table I see the value of 0.9207 which rounds
nicely to 0.921, which was the answer in the previous section.
\end{quote}

\textbf{Part (d):} \(P[\sum_{i = 1}^{16} Z_i^2 < 32]\)

\begin{quote}
For \(Y = \sum_{i = 1}^{16} Z_i^2, Y \sim \chi^2(16)\).
\end{quote}

\begin{Shaded}
\begin{Highlighting}[]
\CommentTok{\# Set variables}
\NormalTok{degrees\_of\_freedom }\OtherTok{\textless{}{-}} \DecValTok{16}
\NormalTok{q }\OtherTok{\textless{}{-}} \DecValTok{32}
\CommentTok{\# Calculate probability}
\NormalTok{prob }\OtherTok{\textless{}{-}} \FunctionTok{pchisq}\NormalTok{(q, degrees\_of\_freedom)}
\CommentTok{\# Print probability}
\FunctionTok{print}\NormalTok{(}\FunctionTok{paste}\NormalTok{(}\StringTok{"The probability that the sample mean is less than 0.5 is"}\NormalTok{, }\FunctionTok{round}\NormalTok{(prob, }\DecValTok{3}\NormalTok{)))}
\end{Highlighting}
\end{Shaded}

\begin{verbatim}
## [1] "The probability that the sample mean is less than 0.5 is 0.99"
\end{verbatim}

\textbf{Part (e):} \(P[\sum_{i = 1}^{16} (Z_i - \bar{Z})^2 < 25]\)

\begin{quote}
For \(Y = \sum_{i = 1}^{16} (Z_i - \bar{Z})^2, Y \sim \chi^2(15)\).
\end{quote}

\begin{Shaded}
\begin{Highlighting}[]
\CommentTok{\# Set variables}
\NormalTok{degrees\_of\_freedom }\OtherTok{\textless{}{-}} \DecValTok{15}
\NormalTok{q }\OtherTok{\textless{}{-}} \DecValTok{25}
\CommentTok{\# Calculate probability}
\NormalTok{prob }\OtherTok{\textless{}{-}} \FunctionTok{pchisq}\NormalTok{(q, degrees\_of\_freedom)}
\CommentTok{\# Print probability}
\FunctionTok{print}\NormalTok{(}\FunctionTok{paste}\NormalTok{(}\StringTok{"The probability that the sample mean is less than 0.5 is"}\NormalTok{, }\FunctionTok{round}\NormalTok{(prob, }\DecValTok{3}\NormalTok{)))}
\end{Highlighting}
\end{Shaded}

\begin{verbatim}
## [1] "The probability that the sample mean is less than 0.5 is 0.95"
\end{verbatim}

\hypertarget{assume-that-z-v_1-and-v_2-are-independent-random-variables-with-z-sim-n01-v_1-sim-chi25-and-v_2-sim-chi29.}{%
\subsubsection{\texorpdfstring{8.18: Assume that \(Z\), \(V_1\), and
\(V_2\) are independent random variables with
\(Z \sim N(0,1), V_1 \sim \chi^2(5)\), and
\(V_2 \sim \chi^2(9)\).}{8.18: Assume that Z, V\_1, and V\_2 are independent random variables with Z \textbackslash sim N(0,1), V\_1 \textbackslash sim \textbackslash chi\^{}2(5), and V\_2 \textbackslash sim \textbackslash chi\^{}2(9).}}\label{assume-that-z-v_1-and-v_2-are-independent-random-variables-with-z-sim-n01-v_1-sim-chi25-and-v_2-sim-chi29.}}

Find the following:

\textbf{Part (a):} \(P[V_1 + V_2 < 8.6]\)

\begin{quote}
For \(Y = V_1 + V_2, Y \sim \chi^2(14)\).
\end{quote}

\begin{Shaded}
\begin{Highlighting}[]
\CommentTok{\# Set variables}
\NormalTok{degrees\_of\_freedom }\OtherTok{\textless{}{-}} \DecValTok{14}
\NormalTok{q }\OtherTok{\textless{}{-}} \FloatTok{8.6}
\CommentTok{\# Calculate probability}
\NormalTok{prob }\OtherTok{\textless{}{-}} \FunctionTok{pchisq}\NormalTok{(q, degrees\_of\_freedom)}
\CommentTok{\# Print probability}
\FunctionTok{print}\NormalTok{(}\FunctionTok{paste}\NormalTok{(}\StringTok{"The probability that the sample mean is less than 0.5 is"}\NormalTok{, }\FunctionTok{round}\NormalTok{(prob, }\DecValTok{3}\NormalTok{)))}
\end{Highlighting}
\end{Shaded}

\begin{verbatim}
## [1] "The probability that the sample mean is less than 0.5 is 0.144"
\end{verbatim}

\textbf{Part (b):} \(P[Z/\sqrt{V_1/5} < 2.015]\)

For \(Y = Z/\sqrt{V_1/5}, Y \sim t(5)\).

\begin{Shaded}
\begin{Highlighting}[]
\CommentTok{\# Set variables}
\NormalTok{degrees\_of\_freedom }\OtherTok{\textless{}{-}} \DecValTok{5}
\NormalTok{q }\OtherTok{\textless{}{-}} \FloatTok{2.015}
\CommentTok{\# Calculate probability}
\NormalTok{prob }\OtherTok{\textless{}{-}} \FunctionTok{pt}\NormalTok{(q, degrees\_of\_freedom)}
\CommentTok{\# Print probability}
\FunctionTok{print}\NormalTok{(}\FunctionTok{paste}\NormalTok{(}\StringTok{"The probability that the sample mean is less than"}\NormalTok{, q, }\StringTok{"is"}\NormalTok{, }\FunctionTok{round}\NormalTok{(prob, }\DecValTok{3}\NormalTok{)))}
\end{Highlighting}
\end{Shaded}

\begin{verbatim}
## [1] "The probability that the sample mean is less than 2.015 is 0.95"
\end{verbatim}

\textbf{Part (c):} \(P[Z > 0.611 \sqrt{V_2}]\)

\begin{quote}
We can rewrite this as \(1 - P[Z/\sqrt{V_2/9} < 1.833]\). For
\(Y = Z/\sqrt{V_2/9}, Y \sim t(9)\).
\end{quote}

\begin{Shaded}
\begin{Highlighting}[]
\CommentTok{\# Set variables}
\NormalTok{degrees\_of\_freedom }\OtherTok{\textless{}{-}} \DecValTok{9}
\NormalTok{q }\OtherTok{\textless{}{-}} \FloatTok{1.833}
\CommentTok{\# Calculate probability}
\NormalTok{prob }\OtherTok{\textless{}{-}} \DecValTok{1} \SpecialCharTok{{-}} \FunctionTok{pt}\NormalTok{(q, degrees\_of\_freedom)}
\CommentTok{\# Print probability}
\FunctionTok{print}\NormalTok{(}\FunctionTok{paste}\NormalTok{(}\StringTok{"The probability that the sample mean is less than"}\NormalTok{, q, }\StringTok{"is"}\NormalTok{, }\FunctionTok{round}\NormalTok{(prob, }\DecValTok{3}\NormalTok{)))}
\end{Highlighting}
\end{Shaded}

\begin{verbatim}
## [1] "The probability that the sample mean is less than 1.833 is 0.05"
\end{verbatim}

\textbf{Part (d):} \(P[V_1/V_2 < 1.450]\)

\begin{quote}
Let's also rewrite this one as
\(P[(V_1/V_2) ((1/5)/(1/9)) < 1.450((1/5)/(1/9)) ] = P[(V_1/5)/(V_2/9) < 2.61 ]\)
\end{quote}

\begin{quote}
For \(Y = (V_1/5)/(V_2/9), Y \sim F(5,9)\).
\end{quote}

\begin{Shaded}
\begin{Highlighting}[]
\CommentTok{\# Set variables}
\NormalTok{degrees\_of\_freedom\_1 }\OtherTok{\textless{}{-}} \DecValTok{5}
\NormalTok{degrees\_of\_freedom\_2 }\OtherTok{\textless{}{-}} \DecValTok{9}
\NormalTok{q }\OtherTok{\textless{}{-}} \FloatTok{2.61}
\CommentTok{\# Calculate probability}
\NormalTok{prob }\OtherTok{\textless{}{-}} \FunctionTok{pf}\NormalTok{(q, degrees\_of\_freedom\_1, degrees\_of\_freedom\_2)}
\CommentTok{\# Print probability}
\FunctionTok{print}\NormalTok{(}\FunctionTok{paste}\NormalTok{(}\StringTok{"The probability that the sample mean is less than"}\NormalTok{, q, }\StringTok{"is"}\NormalTok{, }\FunctionTok{round}\NormalTok{(prob, }\DecValTok{3}\NormalTok{)))}
\end{Highlighting}
\end{Shaded}

\begin{verbatim}
## [1] "The probability that the sample mean is less than 2.61 is 0.9"
\end{verbatim}

\textbf{Part (e):} The value \(b\) such that
\(P[\frac{V_1}{V_1 + V_2} < b] = 0.9\)

Let \(X = V_1 + V_2, X \sim \chi^2(14)\).

Let's now rewrite the equation as follows:

\begin{align*}
&P[\frac{V_1}{X} < b] = 0.9\\
&\implies P[\frac{V_1}{X} \frac{1/5}{1/14} < b\frac{1/5}{1\14}] = 0.9\\
&\text{Let } Y = \frac{V_1/5}{X/14}\\
\implies Y \sim F(5, 14)
\end{align*}

\begin{Shaded}
\begin{Highlighting}[]
\CommentTok{\# Set variables}
\NormalTok{degrees\_of\_freedom\_1 }\OtherTok{\textless{}{-}} \DecValTok{5}
\NormalTok{degrees\_of\_freedom\_2 }\OtherTok{\textless{}{-}} \DecValTok{14}
\NormalTok{p }\OtherTok{\textless{}{-}} \FloatTok{0.9}
\CommentTok{\# Calculate quantile}
\NormalTok{q }\OtherTok{\textless{}{-}} \FunctionTok{qf}\NormalTok{(p, degrees\_of\_freedom\_1, degrees\_of\_freedom\_2)}
\CommentTok{\# Solve for b}
\NormalTok{b }\OtherTok{\textless{}{-}}\NormalTok{ q }\SpecialCharTok{*}\NormalTok{ ((}\DecValTok{1}\SpecialCharTok{/}\DecValTok{14}\NormalTok{)}\SpecialCharTok{/}\NormalTok{(}\DecValTok{1}\SpecialCharTok{/}\DecValTok{5}\NormalTok{))}
\CommentTok{\# Print b}
\FunctionTok{print}\NormalTok{(}\FunctionTok{paste}\NormalTok{(}\StringTok{"b ="}\NormalTok{, }\FunctionTok{round}\NormalTok{(b, }\DecValTok{3}\NormalTok{)))}
\end{Highlighting}
\end{Shaded}

\begin{verbatim}
## [1] "b = 0.824"
\end{verbatim}

\end{document}
